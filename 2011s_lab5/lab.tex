% ex: tw=74 
% ex: tw=74 ts=2 sw=2 noet sts=2
\documentclass[10pt]{article}
\usepackage[utf8]{inputenc}
\usepackage{amsmath}
\usepackage{xkeyval}
\usepackage[bookmarksnumbered,frenchlinks]{hyperref}
%\hypersetup{pdfborder=0 0 0}
\usepackage{multirow}
\usepackage[english]{babel}
\usepackage{fullpage}
\usepackage{tabulary}
\usepackage{tabularx}
%\usepackage{natbib}
\usepackage[all]{hypcap}
\usepackage{hyperref}
\usepackage{framed}
\usepackage{fullpage}
\usepackage{graphicx}
\usepackage{listings}
\usepackage{subfig}
\usepackage{verbatim}
\usepackage{float}
%\floatstyle{boxed} 
%\restylefloat{figure}
\title{\textbf{Lab 5} \\
Ex 2: Act 3; Ex 4: Act 1.1, 2, 3, 4; Ex 5: Act 1.1}
\author{Cody Schafer}
\date{April 29, 2011}
\begin{document}
\maketitle

\section{Introduction}
% An Introduction containing a brief description of the lab

This lab explored the construction and electrical behavior of CMOS logic
(via constructing a few CMOS gates an measuring transition times).
Additionally, regenerative logic circuits (latches and flip-flops) 
were constructed using MOSFETS. Some of these circuits were clocked, and
thus a clock circuit was constructed. Lastly, a CMOS static ram cell was
constructed and read and writing to it was examined.


\section{Activities}
% A description of each activity and what was accomplished

\subsection{Experiment 2, Activity 3}

A CMOS NAND gate, CMOS NOR gate, and CMOS XOR gate were constructed. For
each of these gates, various rise and fall times of the output were
recorded.

\subsection{Experiment 4, Activity 1.1, 2, 3, 4}

A SR NOR latch, a SR NAND latch, a clocked JK flip-flop (using the NAND
latch), and a clocked SR NOR latch were constructed and had their truth
tables verified. For verifying the clocked logic circuits, a clock circuit
which emitted a controlled pulse from a button press was constructed from
BJTs.

\subsection{Experiment 5, Activity 1.1}

A CMOS static ram cell was constructed. A write of both a 1 and a 0 into
the cell was preformed, as was a read of both a 1 and a 0.

\section{Data}
% Any data, tables, or graphs developed in the course of the lab

\subsection{Experiment 2}

\begin{table}[H]
	\centering
	\begin{tabular}{ c|c|c }
	Transitions & rise or fall & Time \\ \hline
	$A:0 \rightarrow 1; B:0 \rightarrow 1$ & fall & 157 nS \\
	$A:0 \rightarrow 1; B:1 \rightarrow 1$ & rise & 241 nS \\
	$A:0 \rightarrow 1; B:1 \rightarrow 0$ & rise & 128 nS \\
	\end{tabular}
	\label{tbl:nand}
	\caption{NAND gate transition times (CMOS)}
\end{table}

\begin{table}[H]
	\centering
	\begin{tabular}{ c|c|c }
	Transitions & rise or fall & Time \\ \hline
	$A:1 \rightarrow 0; B:1 \rightarrow 0; C:1 \rightarrow 0$ & rise &
	220nS \\
	$A:0 \rightarrow 1; B:0 \rightarrow 0; C:0 \rightarrow 0$ & fall &
	107nS \\
	$A:0 \rightarrow 1; B:0 \rightarrow 1; C:0 \rightarrow 1$ & fall &
	17.6nS \\
	\end{tabular}
	\label{tbl:nor}
	\caption{NOR gate transition times (CMOS)}
\end{table}

\begin{table}[H]
	\centering
	\begin{tabular}{ c|c|c }
	Transitions & rise or fall & Time \\ \hline
	$A:1 \rightarrow 0; B:0 \rightarrow 0$ & rise &	141nS \\
	$A:0 \rightarrow 1; B:1 \rightarrow 1$ & fall & 571nS \\
	$A:0 \rightarrow 0; B:1 \rightarrow 0$ & rise &	272nS \\
	$A:0 \rightarrow 0; B:0 \rightarrow 1$ & fall &	133nS \\
	\end{tabular}
	\label{tbl:xor}
	\caption{XOR gate transition times (CMOS)}
\end{table}

\subsection{Experiment 4}

No data of significance recorded, truth tables merely verified (as
demanded by the lab manual). All circuits (SR latch via NOR, SR latch via
NAND, JK Filp-Flop, and Clocked SR Flip-Flop) produced the expected
output.

\subsection{Experiment 5}

\begin{table}[H]
	\centering
	\begin{tabular}{ c|c|c }
	Action & A (volts) & B (volts) \\ \hline
	Store a 1 (5V) in the ram cell, verify. & 4.969 & 0 \\
	Store a 0 (0V) in the ram cell, verify. & 0     & 4.972 \\
	Read the ram cell when it contains a logic 1 & 4.94 & 0.0775 \\
	Read the ram cell when it contains a logic 0 & 0.129 & 5.06 \\
	\end{tabular}
	\label{tbl:ram}
	\caption{Static Ram cell voltage levels}
\end{table}

\section{Discussion}

\subsection{Experiment 2}

From the data collected within this experiment, It can be seen that
different inputs transitioning for the same CMOS logic circuit can cause
changes in the rise and fall times of the outputs. In particular, it is
noted that the distinctions in rise and fall times depend on how many
inputs are transitioning. We can note that in the case of the nor gate,
when all the inputs go high the fall time is significantly lower than the
fall time when only one of the inputs transition to the high state. This
is most likely caused by the much larger current available to charge the
output capacitances (3 transistors are enabled while the output is
transitioning rather than 1).

In the XOR gate's case, as the design varies from the standard reorganized
inverting gates in the NOR and NAND, particularly the use of two
complimenting MOS gates as a type of disable-able buffer causes significant
variance in the rise and fall times for different input patterns. In
addition, it powers one of the inverters contained within it off of an
input line. From the data recorded, the transition of $A:0 \rightarrow 1;
B:1 \rightarrow 1$, where in the inverter is initially lacking in a source
(as A is initially low) requires the largest time to transition, indicating
that if one wished to redesign this gate to reduce the transition times
constructing it in a manner that did not have the source of an
inverter attached to an input would be one of the items to initially
target.


\subsection{Experiment 4}

Each of the circuits constructed within this experiment contain at least a
single feed back from further in the logic, allowing the output of the
circuit to, in a sense, reinforce the output of the circuit (as this
statement appears, the reinforcing is also cyclic). Unlike the static ram
cell constructed in experiment 5, these circuits are structured such that
a minimal amount of current may be drawn when attempting to change the
circuit's state.

\subsection{Experiment 5}

The constructed static ram cell was composed of 2 CMOS inverters fed back into
one another (a simple bistable multi-vibrator design) with additional
attached lines to allow setting of the latch state to either the high or
low values. Additional capacitances were added for load simulation
purposes (typically the lines drive a large number of cells) but it
appears that these loads do not accurately (or simply don't consistently)
model those they are attempting to simulate. The lack of a drop of
significant size while doing the reads of the ram cell, when it was stated
that this was expected, indicated that in reality a much larger leakage
current exists in the capacitive load. In this experiment's case it
appears that the capacitors remained charged to the same voltage that is
stored in the ram cell. Indeed, as the ram cell will constantly stabilize
itself to 5V, the lack of a constant drop for all times is expected.
Instead, the circuit should display (in the worst case, with the load
capacitances holding charge opposite of that stored in the ram cell) a
small initial drop in the voltage toward some intermediate value (~2.5)
prior to recovery to either 0 or 5V, depending on the value stored in the
ram cell.

% The SPICE simulation (the prelab)
%\section{Spice Simulations}
%

\section{Spice Simulations}

\subsection{TIP31 Transistor}
	\begin{figure}[hb!]
		\centering
		\includegraphics[height=3.5in]{tip31_1_01.eps}
		\caption{TIP31, Act 1}
		\label{fig:t1}
	\end{figure}

	\begin{framed}
	\lstinputlisting{tip31_1_01.spice.gen}
	\end{framed}

	\begin{figure}[hb!]
		\centering
		\includegraphics[height=3.5in]{tip31_2_02.eps}
		\caption{TIP31, Act 2}
		\label{fig:t2}
	\end{figure}

	\begin{framed}
	\lstinputlisting{tip31_2_02.spice.gen}
	\end{framed}
	
	\begin{figure}[hb!]
		\centering
		\includegraphics[height=3.5in]{tip31_2_03.eps}
		\caption{TIP31, Act 2}
		\label{fig:t3}
	\end{figure}

	\begin{framed}
	\lstinputlisting{tip31_2_03.spice.gen}
	\end{framed}


	\begin{figure}[hb!]
		\centering
		\includegraphics[height=3.5in]{tip31_2_04.eps}
		\caption{TIP31, Act 2}
		\label{fig:t4}
	\end{figure}

	\begin{framed}
	\lstinputlisting{tip31_2_04.spice.gen}
	\end{framed}

	\begin{figure}[hb!]
		\centering
		\includegraphics[height=3.5in]{tip31_3_05.eps}
		\caption{TIP31, Act 3}
		\label{fig:t5}
	\end{figure}

	\begin{framed}
	\lstinputlisting{tip31_3_05.spice.gen}
	\end{framed}
	
	\begin{figure}[hb!]
		\centering
		\includegraphics[height=3.5in]{tip31_3_06.eps}
		\caption{TIP31, Act 3}
		\label{fig:t6}
	\end{figure}

	\begin{framed}
	\lstinputlisting{tip31_3_06.spice.gen}
	\end{framed}

	\begin{figure}[hb!]
		\centering
		\includegraphics[height=3.5in]{tip31_3_07.eps}
		\caption{TIP31, Act 3}
		\label{fig:t7}
	\end{figure}

	\begin{framed}
	\lstinputlisting{tip31_3_07.spice.gen}
	\end{framed}

	\begin{figure}[hb!]
		\centering
		\includegraphics[height=3.5in]{tip31_4_08.eps}
		\caption{TIP31, Act 4}
		\label{fig:t8}
	\end{figure}

	\begin{framed}
	\lstinputlisting{tip31_4_08.spice.gen}
	\end{framed}


\subsection{2N3904 Transistor}


	\begin{figure}[hb!]
		\centering
		\includegraphics[height=3.5in]{2n3904_1_09.eps}
		\caption{2N3904, Act 1}
		\label{fig:n1}
	\end{figure}

	\begin{framed}
	\lstinputlisting{2n3904_1_09.spice.gen}
	\end{framed}

	\begin{figure}[hb!]
		\centering
		\includegraphics[height=3.5in]{2n3904_2_10.eps}
		\caption{2N3904, Act 2}
		\label{fig:n2}
	\end{figure}

	\begin{framed}
	\lstinputlisting{2n3904_2_10.spice.gen}
	\end{framed}
	
	\begin{figure}[hb!]
		\centering
		\includegraphics[height=3.5in]{2n3904_2_11.eps}
		\caption{2N3904, Act 2}
		\label{fig:n3}
	\end{figure}

	\begin{framed}
	\lstinputlisting{2n3904_2_11.spice.gen}
	\end{framed}


	\begin{figure}[hb!]
		\centering
		\includegraphics[height=3.5in]{2n3904_2_12.eps}
		\caption{2N3904, Act 2}
		\label{fig:n4}
	\end{figure}

	\begin{framed}
	\lstinputlisting{2n3904_2_12.spice.gen}
	\end{framed}

	\begin{figure}[hb!]
		\centering
		\includegraphics[height=3.5in]{2n3904_3_13.eps}
		\caption{2N3904, Act 3}
		\label{fig:n5}
	\end{figure}

	\begin{framed}
	\lstinputlisting{2n3904_3_13.spice.gen}
	\end{framed}
	
	\begin{figure}[hb!]
		\centering
		\includegraphics[height=3.5in]{2n3904_3_14.eps}
		\caption{2N3904, Act 3}
		\label{fig:n6}
	\end{figure}

	\begin{framed}
	\lstinputlisting{2n3904_3_14.spice.gen}
	\end{framed}

	\begin{figure}[hb!]
		\centering
		\includegraphics[height=3.5in]{2n3904_3_15.eps}
		\caption{2N3904, Act 3}
		\label{fig:n7}
	\end{figure}

	\begin{framed}
	\lstinputlisting{2n3904_3_15.spice.gen}
	\end{framed}

	\begin{figure}[hb!]
		\centering
		\includegraphics[height=3.5in]{2n3904_4_16.eps}
		\caption{2N3904, Act 4}
		\label{fig:n8}
	\end{figure}

	\begin{framed}
	\lstinputlisting{2n3904_4_16.spice.gen}
	\end{framed}


\end{document}
