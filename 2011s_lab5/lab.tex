% ex: tw=74 
% ex: tw=74 ts=2 sw=2 noet sts=2
\documentclass[10pt]{article}
\usepackage[utf8]{inputenc}
\usepackage{amsmath}
\usepackage{xkeyval}
\usepackage[bookmarksnumbered,frenchlinks]{hyperref}
%\hypersetup{pdfborder=0 0 0}
\usepackage{multirow}
\usepackage[english]{babel}
\usepackage{fullpage}
\usepackage{tabulary}
\usepackage{tabularx}
%\usepackage{natbib}
\usepackage[all]{hypcap}
\usepackage{hyperref}
\usepackage{framed}
\usepackage{fullpage}
\usepackage{graphicx}
\usepackage{listings}
\usepackage{subfig}
\usepackage{verbatim}
\usepackage{float}
%\floatstyle{boxed} 
%\restylefloat{figure}
\title{\textbf{Lab 5} \\
Ex 2: Act 3; Ex 4: Act 1.1, 2, 3, 4; Ex 5: Act 1.1}
\author{Cody Schafer}
\date{April 29, 2011}
\begin{document}
\maketitle

\section{Introduction}
% An Introduction containing a brief description of the lab

This lab explored the construction and electrical behavior of CMOS logic
(via constructing a few CMOS gates an measuring transisition times).
Additionally, regenerative logic circuits (latches and flip-flops) 
were constructed using MOSFETS. Some of these circuits were clocked, and
thus a clock circuit was constructed. Lastly, a CMOS static ram cell was
constructed and read and writing to it was examined.


\section{Activities}
% A description of each activity and what was accomplished

\subsection{Experiment 2, Activity 3}

A CMOS NAND gate, CMOS NOR gate, and CMOS XOR gate were constructed. For
each of these gates, various rise and fall times of the output were
recoreded.

\subsection{Experiment 4, Activity 1.1, 2, 3, 4}

A SR NOR latch, a SR NAND latch, a clocked JK flip-flop (using the NAND
latch), and a clocked SR NOR latch were constructed and had their truth
tables verified. For verifying the clocked logic circuits, a clock circuit
which emited a controlled pulse from a button press was constructed from
BJTs.

\subsection{Experiment 5, Activity 1.1}

A CMOS static ram cell was constructed. A write of both a 1 and a 0 into
the cell was preformed, as was a read of both a 1 and a 0.

\section{Data}
% Any data, tables, or graphs developed in the course of the lab

\subsection{Experiment 2}

\begin{table}[H]
	\centering
	\begin{tabular}{ c|c|c }
	Transitions & rise or fall & Time \\ \hline
	$A:0 \rightarrow 1; B:0 \rightarrow 1$ & fall & 157 nS \\
	$A:0 \rightarrow 1; B:1 \rightarrow 1$ & rise & 241 nS \\
	$A:0 \rightarrow 1; B:1 \rightarrow 0$ & rise & 128 nS \\
	\end{tabular}
	\label{tbl:nand}
	\caption{NAND gate transisiton times (CMOS)}
\end{table}

\begin{table}[H]
	\centering
	\begin{tabular}{ c|c|c }
	Transitions & rise or fall & Time \\ \hline
	$A:1 \rightarrow 0; B:1 \rightarrow 0; C:1 \rightarrow 0$ & rise &
	220nS \\
	$A:0 \rightarrow 1; B:0 \rightarrow 0; C:0 \rightarrow 0$ & fall &
	107nS \\
	$A:0 \rightarrow 1; B:0 \rightarrow 1; C:0 \rightarrow 1$ & fall &
	17.6nS \\
	\end{tabular}
	\label{tbl:nor}
	\caption{NOR gate transisiton times (CMOS)}
\end{table}

\begin{table}[H]
	\centering
	\begin{tabular}{ c|c|c }
	Transitions & rise or fall & Time \\ \hline
	$A:1 \rightarrow 0; B:0 \rightarrow 0$ & rise &	141nS \\
	$A:0 \rightarrow 1; B:1 \rightarrow 1$ & fall & 571nS \\
	$A:0 \rightarrow 0; B:1 \rightarrow 0$ & rise &	272nS \\
	$A:0 \rightarrow 0; B:0 \rightarrow 1$ & fall &	133nS \\
	\end{tabular}
	\label{tbl:xor}
	\caption{XOR gate transisiton times (CMOS)}
\end{table}

\subsection{Experiment 4}

No data of signifigance recorded, truth tables merely verified (as
demanded by the lab manual). All circuits (SR latch via NOR, SR latch via
NAND, JK Filp-Flop, and Clocked SR Flip-Flop) produced the expected
output.

\subsection{Experiment 5}

\begin{table}[H]
	\centering
	\begin{tabular}{ c|c|c }
	Action & A (volts) & B (volts) \\ \hline
	Store a 1 (5V) in the ram cell, verify. & 4.969 & 0 \\
	Store a 0 (0V) in the ram cell, verify. & 0     & 4.972 \\
	Read the ram cell when it contains a logic 1 & 4.94 & 0.0775 \\
	Read the ram cell when it contains a logic 0 & 0.129 & 5.06 \\
	\end{tabular}
	\label{tbl:ram}
	\caption{Static Ram cell voltage levels}
\end{table}

\section{Discussion}

\subsection{Experiment 2}

\subsection{Experiment 4}

\subsection{Experiment 5}

The constructed static ram cell was composed of 2 CMOS inverters fed back into
one another (a simple bistable multivibrator design) with additional
attached lines to allow setting of the latch state to either the high or
low values. Additional capacitances were added for load simulation
purposes (typically the lines drive a large number of cells) but it
appears that these loads do not accurately (or simply don't consistently)
model those they are attempting to simulate. The lack of a drop of
signifigant size while doing the reads of the ram cell, when it was stated
that this was expected, indicated that in reality a much larger leakage
current exsists in the capacitive load. In this experiment's case it
appears that the capacitors remained charged to the same voltage that is
stored in the ram cell. Indeed, as the ram cell will constantly stablize
itself to 5v, the lack of a constant drop for all times is expected.
Instead, the circuit should display (in the worst case, with the load
capacitances holding charge opposite of that stored in the ram cell) a
small inital drop in the voltage toward some intermediate value (~2.5)
prior to recovery to either 0 or 5v, depending on the value stored in the
ram cell.

% The SPICE simulation (the prelab)
%\section{Spice Simulations}
%
\begin{figure}[H]
	\centering
	\includegraphics[width=4.5in]{p1.eps}
	\caption{Problem 1}
	\label{fig:1}
\end{figure}

\begin{framed}
	\lstinputlisting{p1.spice}
\end{framed}


\begin{figure}[H]
	\centering
	\includegraphics[width=4.5in]{p2.eps}
	\caption{Problem 2}
	\label{fig:2}
\end{figure}

\begin{framed}
	\lstinputlisting{p2.spice}
\end{framed}


\begin{figure}[H]
	\centering
	\includegraphics[width=4.5in]{p3.eps}
	\caption{Problem 3}
	\label{fig:3}
\end{figure}

\begin{framed}
	\lstinputlisting{p3.spice}
\end{framed}


\begin{figure}[H]
	\centering
	\includegraphics[width=4.5in]{p4.eps}
	\caption{Problem 4}
	\label{fig:4}
\end{figure}

\begin{framed}
	\lstinputlisting{p4.spice}
\end{framed}


\begin{figure}[H]
	\centering
	\includegraphics[width=4.5in]{p5.eps}
	\caption{Problem 5}
	\label{fig:5}
\end{figure}

\begin{framed}
	\lstinputlisting{p5.spice}
\end{framed}


\begin{figure}[H]
	\centering
	\includegraphics[width=4.5in]{p6.eps}
	\caption{Problem 6}
	\label{fig:6}
\end{figure}

\begin{framed}
	\lstinputlisting{p6.spice}
\end{framed}



\end{document}
