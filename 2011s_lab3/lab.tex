% ex: tw=74 
% ex: tw=74 ts=2 sw=2 noet sts=2
\documentclass[10pt]{article}
\usepackage[utf8]{inputenc}
\usepackage{amsmath}
\usepackage{xkeyval}
\usepackage[bookmarksnumbered,frenchlinks]{hyperref}
%\hypersetup{pdfborder=0 0 0}
\usepackage{multirow}
\usepackage[english]{babel}
\usepackage{fullpage}
\usepackage{tabulary}
\usepackage{tabularx}
%\usepackage{natbib}
\usepackage[all]{hypcap}
\usepackage{hyperref}
\usepackage{framed}
\usepackage{fullpage}
\usepackage{graphicx}
\usepackage{listings}
\usepackage{subfig}
\usepackage{verbatim}
\usepackage{float}
\usepackage{datatool}
%\floatstyle{boxed} 
%\restylefloat{figure}

\begingroup
\catcode`\ 12
\gdef\DTLsettabseparator{%
\catcode`\ 12
\DTLsetseparator{ }%
}
\endgroup


\title{Experiment 3}
\author{Cody Schafer}
\date{\today}
\begin{document}
\maketitle
\section{Introduction}
% An Introduction containing a brief description of the lab

In this lab the experimentors examine the construction of ECL and TTL
gates as well as the approximate experimental determination of $V_{OH}$,
$V_{OL}$, $V_{IH}$, $V_{IL}$, $NM_{H}$, and $NM_{L}$. For the TTL
circuits, both NAND and NOR gates were constructed. For the ECL case a
combination OR and NOR gate was utilized.

\section{Activities}

\subsection{TTL 2-input NAND}

The 2 input NAND of TTL implimentation was constructed and verified. Then
one of the inputs was fixed high while the other was swept from 0 to 3v in
0.2V steps. All of the constants mentioned above were determined as each
of the data points was recorded. An ampmeter was then connected to the
collector and then high emitter of the input transistor to determine the
$I_{IL}$ and $I_{IH}$ respectivly.

\subsection{TTL 2-input NOR}

Like the previous activity, this circuit was constructed and verified. The
DC sweep was then preformed and the data from this recorded. Instead of
fixing the second input high (at 5v), it was set low (0v). Current
measurements were omited.

\subsection{ECL 2-input NOR}

The same procedure as the previous circuit was utilized. As this activity
shares the same circuit as the following one, all data was recorded in a
single pass. Input voltage was swept from -1.8 to -0.8 with a 0.1V step.
In addition to measuring the output, the base of 2 intermediate gates (1
for this activity and the second related to the following activity)
where also recorded in the data set.

\subsection{ECL 2-input OR}

Construced as part of the previous gate by involving an inverter, this was
subjected to the same verification and analysis as the previous activity.

\section{Data}

\begin{table}[H]
	\centering
	\subfloat[Activity 1, TTL NAND Current recordings]{
		\qquad
		\begin{tabular}{c|c}
			$I_{IH}$ & $I_{IL}$ \\ \hline
			1.476mA  & 1.697mA
		\end{tabular}
	}

	\subfloat[Activity 1, TTL NAND Vin sweep]{
		\qquad
		\begin{tabular}{c|c}
		Vb  & Vo		\\ \hline
		0   & 3.422	\\
		0.2 & 3.452	\\
		0.4	& 3.459	\\
		0.6	& 3.074	\\
		0.8	& 0.231	\\
		1   & 0.167	\\
		1.2	& 0.167	\\
		1.4	& 0.167	\\
		1.6	& 0.166	\\
		1.8	& 0.166	\\
		2.0	& 0.166	\\
		2.2	& 0.166	\\
		2.4	& 0.166	\\
		2.6	& 0.166	\\
		2.8	& 0.165	\\
		3.0	& 0.165
		\end{tabular}
		\label{tbl:act-1}
		\qquad
	}
	\qquad\qquad
	\subfloat[Activity 2, TTL NOR Vin sweep]{
		\qquad
		\begin{tabular}{c|c}
		Vi  & Vo		\\ \hline
		0	  & 4.731	\\
		0.2 &	4.731	\\
		0.4 &	4.731	\\
		0.6	& 4.701	\\
		0.8	& 4.483	\\
		1	  & 4.247	\\
		1.2	& 3.725	\\
		1.4	& 0.072	\\
		1.6	& 0.068	\\
		1.8	& 0.068	\\
		2	  & 0.067	\\
		2.2	& 0.067	\\
		2.4	& 0.067	\\
		2.6	& 0.067	\\
		2.8	& 0.067	\\
		3.0	& 0.067
		\end{tabular}
		\label{tbl:act-2}
		\qquad
	}

	\caption{Activity 1 and 2 data.}
\end{table}

\begin{figure}[H]
	\centering
	\include{include/p1}
	\caption{TTL NAND, input voltage vs output voltage}
	\label{fig:act-1}
\end{figure}
\begin{figure}[H]
	\centering
	\include{include/p2}
	\caption{TTL NOR, input voltage vs output voltage}
	\label{fig:act-2}
\end{figure}
\begin{table}[H]
	\centering
	\begin{tabularx}{0.6\linewidth}{X|X|X|X|X}
	Vi  & Vor   & Vnor  & Vb3   & Vb4		\\ \hline
	-1.8&	-1.472&	-0.706&	-0.776&	-0.004\\
	-1.7&	-1.472&	-0.706&	-0.776&	-0.003\\
	-1.6&	-1.471&	-0.706&	-0.776&	-0.004\\
	-1.5&	-1.471&	-0.706&	-0.776&	-0.004\\
	-1.4&	-1.469&	-0.709&	-0.773&	-0.006\\
	-1.3&	-1.103&	-0.812&	-0.430&	-0.111\\
	-1.2&	-0.984&	-1.136&	-0.315&	-0.605\\
	-1.1&	-0.708&	-1.206&	-0.007&	-0.731\\
	-1.0&	-0.706&	-1.286&	-0.004&	-0.756\\
	-0.9&	-0.705&	-1.468&	-0.004&	-0.778\\
	-0.8&	-0.705&	-1.490&	-0.004&	-0.800
	\end{tabularx}
	\caption{Activity 3 and 4 data.}
	\label{tbl:act-3-4}
\end{table}
\begin{figure}[H]
	\centering
	\include{include/p3}
	\caption{ECL NOR, $V_{B4}$ and output voltage vs input voltage.}
	\label{fig:act-3}
\end{figure}
\begin{figure}[H]
	\centering
	\include{include/p4}
	\caption{ECL OR, $V_{B3}$ and output voltage vs input voltage.}
	\label{fig:act-4}
\end{figure}

\section{Discussion}

\subsection{TTL 2-input NAND Gate}

Examination of the plotted data obtained durring the preformance of the
experiment shows a quickly changing output voltage with respect to the
input voltage, as expected and desired in logic gates. The high output
voltage of a gate, also known as $V_{OH}$, is the worst case output value
which is intended to be read as a high or '1' signal. Similarly, the low
output voltage, or $V_{OL}$, is the worst case output voltage which is
intended to be read as a low or '0' signal. For this activity, the $V_{OH}$ and
$V_{OL}$ were determined to be 3.074 and 0.167 Volts, respectively. These
choices were based upon the stabilization seen in the output voltage
around the prior indicated measurments. It was also determined that
$V_{IH}$ and $V_{IL}$ were 0.6 and 1 Volt, respectively. These were chosen
as when they are inputed into the gate, they result in voltages near the
high and low output voltages, respectively.

The input currents measured were rather large, but still within the
expected range of TTL input currents. These rather large currents, even
when not switching, make TTL a larger power expenditure than some other
logic setups (with ECL exceeding even TTL in power consumption).

Noise margin calculations were preformed, and the resulting $NM_L$ and
$NM_H$ were 0.433 and 2.074 volts, respectively. A noise margin is the
difference between the output and input voltages of the same intended
level, thus $NM_H = V_{OH} - V_{IH}$ and $NM_L = V_{IL} - V_{OL}$.

\subsection{TTL 2-input NOR Gate}

This transfer curve shows a similar responce to the previous activity, as
the circuit uses the same design style. Worth noting, however, is that the
wave form is essentially shifted by some fraction of a volt (~0.5, by eye)
from the waveform produced by activity 1. This can be attributed to the
extra level of transistors present in this gate, which should (by our
general estimation methods) cause a shift of around 0.7v.

The determined $V_{OH}$ and $V_{IH}$ were 4.247 and 1.2 volt, respectively.
The determined $V_{OL}$ and $V_{IL}$ were 0.072 and 1.0 volts,
respectively. Noise margins $NM_H$ and $NM_L$ were calculated as 3.047 and
0.928 volts respectively.

\subsection{ECL 2-input NOR Gate}

The plot generated from the recorded data in this activity shows a closely
tracking $V_{B4}$ vs $V_o$, as expected due to the clamping between base
and emitter to 0.7 Volts. Another interesting feature of this activity's
plot (as well as the following one), is a change in the rate of change in
the outputs near the center of the graph. This is probably caused by the
input voltages crossing the $V_{REF}$ input to the reference transistor.

The determined $V_{OL}$ and $V_{IL}$ were -1.136 and -1.4 volts,
respectively.  The determined $V_{OH}$ and $V_{IH}$ were -0.709 and -1.2
volts, respectively. Noise margins $NM_L$ and $NM_H$ were calculated as
0.264 and 1.11 volts respectively.

\subsection{ECL 2-input OR Gate}

Similar to the above section, $V_{B3}$ closely tracks $V_o$.

The determined $V_{OL}$ and $V_{IL}$ were -1.469 and -1.4 volts,
respectively.  The determined $V_{OH}$ and $V_{IH}$ were -0.984 and -1.2
volts, respectively. Noise margins $NM_L$ and $NM_H$ were calculated as
0.069 and 0.216 volts respectively.

These margins are much lower than the other's encounterd, possibly due to
the use of an addtional inversion within the gate to obtain the desired
logical output.

% The SPICE simulation (the prelab)
\section{Spice Simulations}

\begin{figure}[H]
	\centering
	\includegraphics[width=4.5in]{p1.eps}
	\caption{Problem 1}
	\label{fig:1}
\end{figure}

\begin{framed}
	\lstinputlisting{p1.spice}
\end{framed}


\begin{figure}[H]
	\centering
	\includegraphics[width=4.5in]{p2.eps}
	\caption{Problem 2}
	\label{fig:2}
\end{figure}

\begin{framed}
	\lstinputlisting{p2.spice}
\end{framed}


\begin{figure}[H]
	\centering
	\includegraphics[width=4.5in]{p3.eps}
	\caption{Problem 3}
	\label{fig:3}
\end{figure}

\begin{framed}
	\lstinputlisting{p3.spice}
\end{framed}


\begin{figure}[H]
	\centering
	\includegraphics[width=4.5in]{p4.eps}
	\caption{Problem 4}
	\label{fig:4}
\end{figure}

\begin{framed}
	\lstinputlisting{p4.spice}
\end{framed}


\begin{figure}[H]
	\centering
	\includegraphics[width=4.5in]{p5.eps}
	\caption{Problem 5}
	\label{fig:5}
\end{figure}

\begin{framed}
	\lstinputlisting{p5.spice}
\end{framed}


\begin{figure}[H]
	\centering
	\includegraphics[width=4.5in]{p6.eps}
	\caption{Problem 6}
	\label{fig:6}
\end{figure}

\begin{framed}
	\lstinputlisting{p6.spice}
\end{framed}



\end{document}
