% ex: tw=74 
\documentclass[dvips,10pt]{report}
\usepackage[utf8]{inputenc}
\usepackage{amsmath}
\usepackage{xkeyval}
\usepackage[bookmarksnumbered,frenchlinks]{hyperref}
\hypersetup{pdfborder=0 0 0}
\usepackage{multirow}
\usepackage[english]{babel}
\usepackage{fullpage}
\usepackage{tabulary}
\usepackage{tabularx}
%\usepackage{natbib}
\usepackage[all]{hypcap}
\usepackage{hyperref}
\usepackage{framed}
\usepackage{fullpage}
\usepackage{graphicx}
\usepackage{listings}
\usepackage{float}
\floatstyle{boxed} 
\restylefloat{figure}
\title{\textbf{Experiment 1} \\
	Activity 1}
\author{Cody Schafer}
\date{\today}
\begin{document}
\maketitle

\section{Introduction}
% An Introduction containing a brief description of the lab

In this lab the experimentor was familiarized with the capacative effects
of varieing diodes by measuring physically avaliable timings related to
these capacitances and then calculating them.

\section{Activites}
% A description of each activity and what was accomplished
Activity 1 (the only activity carried out for this lab) features the
determination of the $C_j$ and $C_d$ for the 6A05, 914, and 1N4001 diodes.
Various timing constants were also calculated.

\section{Data}
% Any data, tables, or graphs developed in the course of the lab

\begin{center}
\begin{tabularx}{\textwidth}{X|X|X|X|X|X|X|X|X}
	~ & 
	\multicolumn{5}{c|}{ -10 to 10 Volts} & 
	\multicolumn{3}{c}{ 0 to 10 Volts} \\

	Diode  & $T_D$  & $T_R$  & $T_S$ & $V_b$  & $V_a$   & $T_S$ & $V_b$
	& $V_a$ \\ \hline

	D914   & 93ns   & 356ns  & 12.6ns & 0.6 V & -10 V   & 9.07ns & 0.74
	V & -0.01 V \\

	1N4001 & 91.7ns & 6.4us  & 4.27us & 0.7 V & -10 V   & 25.7us & 0.2
	V & 0 V \\

	6A05   & 120ns  & 4.35us & 21.4us & 0.6 V & -10.1 V & 131us & 0.64
	V & 0.008 V \\
\end{tabularx}
\end{center}

\section{Calculations}
% Any questions posed durring the lab answered

\subsection{Calculation of $C_d$ with an input voltage of -10 to 10.}

\begin{eqnarray}
	I_D(F) = \frac{V_{in}(F) - V_b}{R} \\
	I_D(R) = \frac{V_{in}(R) - V_b}{R} \\
	\tau_T = \frac{T_s}{ln \left( 1 - \frac{I_D(F)}{I_D(R)} \right) } \\
	r_d = \frac{V_T}{I_D(F)} \\
	C_d = \frac{\tau_T}{r_d} \\
\end{eqnarray}


\begin{center}
\begin{tabularx}{\textwidth}{X|X|X|X|X|X}

	Diode  & $I_D(F)$ & $I_D(R)$ & $\tau_T$    & $r_d$ & $C_d$ \\ \hline
	
	D914   & 9.4 mA    & -10.6 mA & 19.846e-9 s & $2.660 \Omega$ &
	7.462e-9 F\\

	1N4001 & 9.3 mA    & -10.7 mA & 6.8227e-6 s & $2.688 \Omega$ & 
	2.538e-6 F\\

	6A05   & 9.4 mA    & -10.6 mA & 33.707e-6 s & $2.660 \Omega$ &
	12.674e-6 F

\end{tabularx}
\end{center}


\subsection{Calculation of $C_j$ with an input voltage of -10 to 10}

\begin{eqnarray}
	I_{R1} = \frac{-10 - 0.7}{1K} = -10.7 mA \\
	I_{R2} = \frac{-10 - (-10)}{1K}	= 0 A	\\
	I_{av} = \frac{I_{R1} + I_{R2}}{2} = -5.35 mA \\
	C_j(av) = \frac{|I_{av}| \times T_r}{\Delta V}\\
	\Delta V = 10.7
	K = \frac{-V_r}{\phi_j} \\
	C_j(0) = \frac{C_j(av) \sqrt{1 + K}}{1.5} \\
	C_j(-10) = \frac{C_j(0)}{1-\frac{V_\Delta}{\phi_d}}
\end{eqnarray}

\begin{center}
\begin{tabularx}{\textwidth}{X|X|X|X}
	Diode  & $C_j(av)$  & $C_j(0)$ & $C_j(-10)$ \\ \hline

	D914   & 178e-12 F  & 463.95e-12 F & 118.66e-12 F \\
	1N4001 & 3.2e-9 F   & 8.34068e-9 F & 2.1333e-9  F \\
	6A05   & 2.175e-9 F & 5.669e-9 F   & 1.45e-9 F

\end{tabularx}
\end{center}

\subsection{Calculation of $C_d$ with an input voltage of 0 to 10}

\begin{center}
\begin{tabularx}{\textwidth}{X|X|X|X|X|X}

	Diode  & $I_D(F)$ & $I_D(R)$ & $\tau_T$   & $r_d$           &
	$C_d$ \\ \hline
	
	D914   & 9.26 mA  & -0.74 mA & 1.875e-9 s & 2.6998 $\Omega$ & 
	694.495e-12 F\\

	1N4001 & 9.3 mA   & -0.7 mA  & 9.664e-6 s & 2.6882 $\Omega$ &
	3.59497e-6 F \\

	6A05   & 9.36 mA  & -0.64 mA & 47.66e-6 s & 2.6709 $\Omega$ &
	17.844e-6 F
\end{tabularx}
\end{center}

\section{Conclution/Summary}
% A brief conclution/summary outlining the theory/concept explored in the lab.

The calculations of both $C_j$ and $C_d$ were based on equations needed
for the class. Notable in the results was the presense of the storage
time, $T_s$, which caused a delay in the fall of the voltage across the
diode. In the case of the large 6a05 diode, a slight slope within the
period of $T_s$ was visible prior to the voltage decrease becomming
obviously exponential in nature. The diode's larger size appeared to be a
contributing factor in causing that result.

The calculated results for the 6a05 diode versus the 1n4001 diode appear
to show some type of error in the measurment of the $T_R$ value, as the
$C_j$'s of the 6a05 were determined to be lower than those of the 1n4001,
which is unexpected. The 6a05 diode was expected to have the highest
capacitance due to it's much larger semiconductor volumes.

% The SPICE simulation (the prelab)

\begin{figure}[H]
	\centering
	\includegraphics[width=4.5in]{p1.eps}
	\caption{Problem 1}
	\label{fig:1}
\end{figure}

\begin{framed}
	\lstinputlisting{p1.spice}
\end{framed}


\begin{figure}[H]
	\centering
	\includegraphics[width=4.5in]{p2.eps}
	\caption{Problem 2}
	\label{fig:2}
\end{figure}

\begin{framed}
	\lstinputlisting{p2.spice}
\end{framed}


\begin{figure}[H]
	\centering
	\includegraphics[width=4.5in]{p3.eps}
	\caption{Problem 3}
	\label{fig:3}
\end{figure}

\begin{framed}
	\lstinputlisting{p3.spice}
\end{framed}


\begin{figure}[H]
	\centering
	\includegraphics[width=4.5in]{p4.eps}
	\caption{Problem 4}
	\label{fig:4}
\end{figure}

\begin{framed}
	\lstinputlisting{p4.spice}
\end{framed}


\begin{figure}[H]
	\centering
	\includegraphics[width=4.5in]{p5.eps}
	\caption{Problem 5}
	\label{fig:5}
\end{figure}

\begin{framed}
	\lstinputlisting{p5.spice}
\end{framed}


\begin{figure}[H]
	\centering
	\includegraphics[width=4.5in]{p6.eps}
	\caption{Problem 6}
	\label{fig:6}
\end{figure}

\begin{framed}
	\lstinputlisting{p6.spice}
\end{framed}



\end{document}
