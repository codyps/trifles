% ex: tw=74 
% ex: tw=74 ts=2 sw=2 noet sts=2
\documentclass[10pt]{article}
\usepackage[utf8]{inputenc}
\usepackage{amsmath}
\usepackage{xkeyval}
\usepackage[bookmarksnumbered,frenchlinks]{hyperref}
%\hypersetup{pdfborder=0 0 0}
\usepackage{multirow}
\usepackage[english]{babel}
\usepackage{fullpage}
\usepackage{tabulary}
\usepackage{tabularx}
%\usepackage{natbib}
\usepackage[all]{hypcap}
\usepackage{hyperref}
\usepackage{framed}
\usepackage{fullpage}
\usepackage{graphicx}
\usepackage{listings}
\usepackage{subfig}
\usepackage{verbatim}
\usepackage{float}
%\floatstyle{boxed} 
%\restylefloat{figure}
\title{\textbf{Lab} 4 \\
NMOS and CMOS inverter}
\author{Cody Schafer}
\date{\today}
\begin{document}
\maketitle
\section{Discussion}

\subsection{Activity 1, NMOS inverter}

In tabulating and graphically representing the input voltage versus output
voltage curves of the NMOS inverter, it's non-symmetric nature became
readily apparent. As only one side of the system is acting in switching
mode, the crisp latching to both a steady output high and output low with
in the same circuit, as seen in CMOS circuits was missing. In the
particular circuit examined (an NMOS inverter) the high output voltage was
clearly held at 1.972 volts, as both of the switching transistors (both
used in pulling down the output) were in cut-off mode (had this been a
PMOS setup, the low output voltage would have been crisp instead). As the
input voltage rose, the low output voltage never stabilized, or even
slowed to the extent seen in CMOS circuits, it continued to drop even as
we exceeded the highest possible input voltage. The extra switching
transistor added on the low half of the gate has the purpose of aiding in
sinking current, as the resistive transistor has the ability to source
more, due to it having it's gate tied high to an optimal input voltage,
while the switching transistors have to cope with a potentially (and
probably) non-optimal input voltage.

The high output voltage of 1.956 causes the experimenter some concern that
one of the switching transistors may have been damaged and failed open, in
essence, turning into a resistive transistor pulling down the output
voltage. Having the high output be somewhat less than half of the expected
value supports this hypothesis if the design of having 2 switching
transistors was chosen to source slightly more than the needed current.
However, the systems response to removing one of the switching
transistors from the circuit appears to contradict this hypothesis of the
reason for failure. Still, some other failure mode for at least one of the
transistors remains probable.

The second transistor additionally appears to aid in reducing the rise and
fall times. This is attributable to the additional current in sinks
allowing the parasitic capacitances to drain/charge at a higher rate.



\subsection{Activity 2, CMOS inverter}

The graphs of data obtained from the tabulations of input and output
voltages from this gate were nearly symmetrical, as expected. Additionally,
the rise and fall times neared equality, and their differences were
significantly lower than the differences in rise and fall times of the
NMOS circuit. The rise and fall times of the CMOS inverter are comparable
to those observed in the NMOS inverter circuit, probably due in a large
part to the use of the same transistor package in both circuits, thus they
have the same transistor characteristics (K) and parasitic capacitances.

Both the CMOS and NMOS inverter showed the change (rise or fall) of the
input voltages to proceed at a much higher rate than the change (fall or
rise) in the output voltages. This occurs due to the output voltage's
dependence on the transition of the input voltage in determining the flow
of current away from capacitances attached to the output as well as the
lack of a serious impediment to the flow of current into the gate
capacitances. Of course, when the $R_G$ was added, we can clearly see an
increase in the time for the input voltage to transition, but the output
voltage transition times were affected as well due to the current flow
into or away from the output being a function of the gate voltage.



% The SPICE simulation (the prelab)
%\section{Spice Simulations}
%

\section{Spice Simulations}

\subsection{TIP31 Transistor}
	\begin{figure}[hb!]
		\centering
		\includegraphics[height=3.5in]{tip31_1_01.eps}
		\caption{TIP31, Act 1}
		\label{fig:t1}
	\end{figure}

	\begin{framed}
	\lstinputlisting{tip31_1_01.spice.gen}
	\end{framed}

	\begin{figure}[hb!]
		\centering
		\includegraphics[height=3.5in]{tip31_2_02.eps}
		\caption{TIP31, Act 2}
		\label{fig:t2}
	\end{figure}

	\begin{framed}
	\lstinputlisting{tip31_2_02.spice.gen}
	\end{framed}
	
	\begin{figure}[hb!]
		\centering
		\includegraphics[height=3.5in]{tip31_2_03.eps}
		\caption{TIP31, Act 2}
		\label{fig:t3}
	\end{figure}

	\begin{framed}
	\lstinputlisting{tip31_2_03.spice.gen}
	\end{framed}


	\begin{figure}[hb!]
		\centering
		\includegraphics[height=3.5in]{tip31_2_04.eps}
		\caption{TIP31, Act 2}
		\label{fig:t4}
	\end{figure}

	\begin{framed}
	\lstinputlisting{tip31_2_04.spice.gen}
	\end{framed}

	\begin{figure}[hb!]
		\centering
		\includegraphics[height=3.5in]{tip31_3_05.eps}
		\caption{TIP31, Act 3}
		\label{fig:t5}
	\end{figure}

	\begin{framed}
	\lstinputlisting{tip31_3_05.spice.gen}
	\end{framed}
	
	\begin{figure}[hb!]
		\centering
		\includegraphics[height=3.5in]{tip31_3_06.eps}
		\caption{TIP31, Act 3}
		\label{fig:t6}
	\end{figure}

	\begin{framed}
	\lstinputlisting{tip31_3_06.spice.gen}
	\end{framed}

	\begin{figure}[hb!]
		\centering
		\includegraphics[height=3.5in]{tip31_3_07.eps}
		\caption{TIP31, Act 3}
		\label{fig:t7}
	\end{figure}

	\begin{framed}
	\lstinputlisting{tip31_3_07.spice.gen}
	\end{framed}

	\begin{figure}[hb!]
		\centering
		\includegraphics[height=3.5in]{tip31_4_08.eps}
		\caption{TIP31, Act 4}
		\label{fig:t8}
	\end{figure}

	\begin{framed}
	\lstinputlisting{tip31_4_08.spice.gen}
	\end{framed}


\subsection{2N3904 Transistor}


	\begin{figure}[hb!]
		\centering
		\includegraphics[height=3.5in]{2n3904_1_09.eps}
		\caption{2N3904, Act 1}
		\label{fig:n1}
	\end{figure}

	\begin{framed}
	\lstinputlisting{2n3904_1_09.spice.gen}
	\end{framed}

	\begin{figure}[hb!]
		\centering
		\includegraphics[height=3.5in]{2n3904_2_10.eps}
		\caption{2N3904, Act 2}
		\label{fig:n2}
	\end{figure}

	\begin{framed}
	\lstinputlisting{2n3904_2_10.spice.gen}
	\end{framed}
	
	\begin{figure}[hb!]
		\centering
		\includegraphics[height=3.5in]{2n3904_2_11.eps}
		\caption{2N3904, Act 2}
		\label{fig:n3}
	\end{figure}

	\begin{framed}
	\lstinputlisting{2n3904_2_11.spice.gen}
	\end{framed}


	\begin{figure}[hb!]
		\centering
		\includegraphics[height=3.5in]{2n3904_2_12.eps}
		\caption{2N3904, Act 2}
		\label{fig:n4}
	\end{figure}

	\begin{framed}
	\lstinputlisting{2n3904_2_12.spice.gen}
	\end{framed}

	\begin{figure}[hb!]
		\centering
		\includegraphics[height=3.5in]{2n3904_3_13.eps}
		\caption{2N3904, Act 3}
		\label{fig:n5}
	\end{figure}

	\begin{framed}
	\lstinputlisting{2n3904_3_13.spice.gen}
	\end{framed}
	
	\begin{figure}[hb!]
		\centering
		\includegraphics[height=3.5in]{2n3904_3_14.eps}
		\caption{2N3904, Act 3}
		\label{fig:n6}
	\end{figure}

	\begin{framed}
	\lstinputlisting{2n3904_3_14.spice.gen}
	\end{framed}

	\begin{figure}[hb!]
		\centering
		\includegraphics[height=3.5in]{2n3904_3_15.eps}
		\caption{2N3904, Act 3}
		\label{fig:n7}
	\end{figure}

	\begin{framed}
	\lstinputlisting{2n3904_3_15.spice.gen}
	\end{framed}

	\begin{figure}[hb!]
		\centering
		\includegraphics[height=3.5in]{2n3904_4_16.eps}
		\caption{2N3904, Act 4}
		\label{fig:n8}
	\end{figure}

	\begin{framed}
	\lstinputlisting{2n3904_4_16.spice.gen}
	\end{framed}


\end{document}
